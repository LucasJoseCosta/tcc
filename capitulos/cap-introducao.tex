%%%% CAPÍTULO 1 - INTRODUÇÃO
%%
%% Deve apresentar uma visão global da pesquisa, incluindo: breve histórico, importância e justificativa da escolha do tema,
%% delimitações do assunto, formulação de hipóteses e objetivos da pesquisa e estrutura do trabalho.

%% Título e rótulo de capítulo (rótulos não devem conter caracteres especiais, acentuados ou cedilha)
\chapter{Introdução}
\label{cha:introducao}

A agricultura familiar representa um pilar fundamental para a segurança alimentar e para a dinâmica socioeconômica do Brasil, destacando-se no Oeste do Paraná, onde o município de Toledo evidencia sua relevância \cite{ibge_censo2017}. Apesar de sua importância, os pequenos produtores rurais enfrentam desafios recorrentes na gestão de suas atividades. Estudos na área de gestão agroindustrial apontam que dificuldades no controle financeiro e de produção são barreiras significativas que impactam diretamente a rentabilidade e a sustentabilidade dessas propriedades \cite{batalha2007gestaoAgroindustrial}.

A crescente digitalização do agronegócio oferece um vasto leque de ferramentas para mitigar esses desafios. Contudo, a escolha da plataforma tecnológica para o desenvolvimento de tais soluções não é trivial. Na engenharia de software, a seleção de tecnologias é uma decisão estratégica que influencia diretamente os custos, o cronograma, a qualidade e a manutenibilidade de um projeto de software \cite{pressman2014engenhariaSoftware}. Uma escolha inadequada pode resultar em aplicações com baixo desempenho, de difícil manutenção ou com alta curva de aprendizado para a equipe de desenvolvimento.

Diante deste cenário, surge a questão central que norteia este trabalho: \textbf{qual tecnologia oferece o melhor balanço entre desempenho, complexidade de implementação e recursos para o desenvolvimento de uma aplicação de gestão agrícola focada no processamento de dados financeiros?} 

Este trabalho propõe uma análise técnica comparativa entre três ecossistemas de desenvolvimento distintos: React Native com TypeScript, C\# com .NET MAUI e Python com a biblioteca PyQt. O estudo de caso será a implementação de um módulo de cálculo de custo e lucratividade por cultura, uma funcionalidade chave de um protótipo concebido como "AgroGestor". Através da implementação desta funcionalidade em cada uma das tecnologias, o trabalho buscará quantificar e analisar métricas de desempenho, uso de recursos, e aspectos qualitativos da experiência de desenvolvimento.

O objetivo geral, portanto, é fornecer uma análise fundamentada que possa guiar futuras decisões de desenvolvimento de software para o agronegócio, comparando os trade-offs de cada plataforma. Espera-se que os resultados contribuam para a comunidade acadêmica e para desenvolvedores de software, oferecendo um panorama claro das vantagens e desvantagens de cada abordagem tecnológica na construção de aplicações de gestão de dados.
