%%%% CAPÍTULO 3 - MATERIAL E MÉTODOS (PODE SER OUTRO TÍTULO DE ACORDO COM O TRABALHO REALIZADO)

\chapter{Metodologia}
\label{cha:metodologia}

Este capítulo apresenta o delineamento metodológico e os procedimentos técnicos adotados para a realização da análise comparativa entre as tecnologias selecionadas. O objetivo é estabelecer um protocolo rigoroso e reprodutível para a coleta e análise de dados quantitativos e qualitativos.

\section{Definição do Problema Computacional}
\label{sec:problema_computacional}

O estudo de caso para esta análise consiste na implementação de um \textbf{"Módulo de Cálculo de Custo e Lucratividade por Cultura"}. A funcionalidade foi selecionada por sua relevância para a gestão agrícola e por sua natureza computacional, que envolve operações de I/O, processamento de dados em memória e renderização de interface gráfica.

\subsection{Especificação da Tarefa}
A tarefa a ser implementada em cada tecnologia é definida pelas seguintes etapas:
\begin{enumerate}
	\item \textbf{Input:} Ler e interpretar um arquivo de dados em formato CSV contendo 10.000 lançamentos financeiros.
	\item \textbf{Processamento:} Mapear os dados planos do CSV para um modelo de dados hierárquico em memória, conforme especificado na Subseção \ref{ssec:modelo_dados}.
	\item \textbf{Output:} Exibir os resultados agregados em uma interface gráfica desktop, apresentando uma tabela ou lista com as colunas: "Cultura", "Total Receitas", "Total Despesas" e "Balanço Final", além de métricas calculadas como o balanço por hectare.
\end{enumerate}

% NOVA SUBSEÇÃO ADICIONADA AQUI
\subsection{Modelo de Dados}
\label{ssec:modelo_dados}

Para adicionar um nível de complexidade e realismo ao problema, foi definido um modelo de dados hierárquico que simula a organização de uma propriedade agrícola. A tarefa computacional central envolve a transformação dos dados de um formato plano (CSV) para este modelo estruturado em memória.

\begin{itemize}
	\item \textbf{Estrutura de Alto Nível: Safra}
	\begin{itemize}
		\item \texttt{identificador}: string (ex: "Safra 2024/2025")
		\item \texttt{culturas}: uma lista de objetos do tipo \texttt{Cultura}
	\end{itemize}
	\item \textbf{Estrutura Intermediária: Cultura}
	\begin{itemize}
		\item \texttt{nome}: string (ex: "Soja")
		\item \texttt{area\_plantada\_hectares}: float
		\item \texttt{lancamentos}: uma lista de objetos do tipo \texttt{LancamentoFinanceiro}
	\end{itemize}
	\item \textbf{Estrutura Base: LancamentoFinanceiro}
	\begin{itemize}
		\item \texttt{id}: int
		\item \texttt{data}: date
		\item \texttt{tipo}: string ('receita' ou 'despesa')
		\item \texttt{categoria}: string (ex: "Insumos", "Mão de Obra", "Venda Cooperativa")
		\item \texttt{descricao}: string
		\item \texttt{valor}: float
	\end{itemize}
\end{itemize}

\section{Ambiente de Teste}
\label{sec:ambiente}

Para garantir a consistência, a imparcialidade e a reprodutibilidade dos resultados, todos os testes de desenvolvimento e desempenho foram executados em um único e consistente ambiente de hardware e software. As especificações exatas da máquina utilizada são detalhadas abaixo.

\begin{itemize}
	\item \textbf{Hardware:}
	\begin{itemize}
		\item Processador: Intel(R) Core(TM) i5-6200U CPU @ 2.30GHz
		\item Memória RAM: 20,0 GB
		\item Armazenamento: Unidade de estado sólido (SSD) SATA de 960 GB
	\end{itemize}
	\vspace{0.2cm} % Adiciona um pequeno espaço vertical
	\item \textbf{Software:}
	\begin{itemize}
		\item Sistema Operacional: Windows 11 Home Single Language (64-bit), Versão 24H2
		\item Versão do Python: 3.13.7
		\item Versão do Node.js: v22.18.0
		\item Versão do .NET SDK: 9.0.304
	\end{itemize}
\end{itemize}

\section{Tecnologias Selecionadas para Análise}
\label{sec:tecnologias_selecionadas}

Foram selecionadas três plataformas tecnológicas que representam diferentes paradigmas no desenvolvimento de aplicações desktop:

\begin{itemize}
	\item \textbf{Python com CustomTkinter:} Representando o desenvolvimento rápido com uma linguagem interpretada, dinamicamente tipada, e uma biblioteca de interface gráfica moderna.
	\item \textbf{Electron com TypeScript e React:} Representando o desenvolvimento de aplicações desktop utilizando tecnologias web, combinando um backend Node.js com uma interface reativa e tipada.
	\item \textbf{C\# com WPF (Windows Presentation Foundation):} Representando a abordagem tradicional para aplicações Windows com uma linguagem compilada, fortemente tipada e um framework de UI maduro do ecossistema .NET.
\end{itemize}


\section{Métricas de Avaliação}
\label{sec:metricas}

A comparação entre as implementações será realizada com base em um conjunto expandido de métricas quantitativas e qualitativas.

\subsection{Métricas Quantitativas}
\begin{itemize}
	\item \textbf{Tempo de Processamento (ms):} Será medido o tempo médio de 10 execuções para o processamento completo do dataset (etapa 2 da tarefa). Será reportada a média e o desvio padrão para avaliar a consistência do desempenho.
	\item \textbf{Tempo de CPU (ms):} Para isolar o custo computacional do tempo de I/O, será medido o tempo de processador consumido pela tarefa, uma métrica mais precisa de eficiência de código.
	\item \textbf{Uso de Memória RAM (MB):} Será monitorado o pico de consumo de memória do processo (Peak Working Set) durante a execução, utilizando as ferramentas de diagnóstico de cada sistema operacional.
	\item \textbf{Tempo de Renderização da UI (ms):} Medida do tempo decorrido entre o fim do processamento dos dados e a renderização completa da tabela de resultados na interface gráfica.
	\item \textbf{Tamanho da Aplicação (MB):} Medida do tamanho final do pacote de distribuição/executável gerado e pronto para o usuário final.
	\item \textbf{Linhas de Código (LoC):} Contabilização das linhas de código escritas pelo desenvolvedor, separando a lógica de negócio da interface de usuário, para avaliar a verbosidade e o esforço de implementação.
	\item \textbf{Número de Dependências:} Contagem de bibliotecas de terceiros necessárias para a implementação de cada solução.
\end{itemize}

\subsection{Métricas Qualitativas}
\begin{itemize}
	\item \textbf{Ergonomia do Desenvolvedor:} Avaliação da experiência de desenvolvimento, incluindo a facilidade de configuração do ambiente, a velocidade do ciclo de build/debug (ex: hot reload vs. recompilação completa) e a clareza das mensagens de erro.
	\item \textbf{Complexidade Arquitetural:} Análise da arquitetura de software que cada tecnologia incentiva (ex: baseada em componentes, MVVM, etc.) e a complexidade para implementar um código limpo e manutenível.
	\item \textbf{Legibilidade e Manutenibilidade do Código:} Comparação da sintaxe e da estrutura do código para avaliar a facilidade de compreensão e modificação futura.
	\item \textbf{Qualidade do Ecossistema:} Análise da maturidade e abrangência do ecossistema de cada tecnologia, incluindo a qualidade da documentação oficial e o suporte da comunidade.
\end{itemize}