%%%% CAPÍTULO 2 - REVISÃO DA LITERATURA (OU REVISÃO BIBLIOGRÁFICA, ESTADO DA ARTE, ESTADO DO CONHECIMENTO)
%%
%% O autor deve registrar seu conhecimento sobre a literatura básica do assunto, discutindo e comentando a informação já publicada.
%% A revisão deve ser apresentada, preferencialmente, em ordem cronológica e por blocos de assunto, procurando mostrar a evolução do tema.
%% Título e rótulo de capítulo (rótulos não devem conter caracteres especiais, acentuados ou cedilha)
\chapter{Referencial te\'orico}\label{cap:referencialTeorico}

Este capítulo apresenta os conceitos fundamentais sobre as tecnologias selecionadas para a análise comparativa. O objetivo é descrever a arquitetura, as principais características, vantagens e desvantagens de cada ecossistema de desenvolvimento.

\section{React Native}
\label{sec:react_native}

O React Native é um framework de código aberto criado pelo Facebook para o desenvolvimento de aplicativos móveis nativos utilizando JavaScript e React. Sua principal proposta de valor é a capacidade de compartilhar uma grande parte do código-base entre as plataformas iOS e Android... 
% Aqui você desenvolverá o texto sobre o React Native, abordando:
% - Arquitetura (Bridge, JSI)
% - Componentização com React
% - Ecossistema (npm, bibliotecas populares como Expo)
% - Vantagens (reuso de código, hot reload) e desvantagens (dependência da bridge, performance em tarefas pesadas)

\section{.NET MAUI}
\label{sec:dotnet_maui}

O .NET Multi-platform App UI (.NET MAUI) é a evolução do Xamarin.Forms, um framework da Microsoft para a criação de aplicações nativas para Windows, macOS, Android e iOS a partir de um único projeto em C\# e XAML. Ele faz parte da plataforma unificada .NET...
% Aqui você desenvolverá o texto sobre o .NET MAUI, abordando:
% - Arquitetura (plataforma unificada .NET, compilação AOT)
% - Linguagem C# e XAML para UI
% - Ecossistema (Visual Studio, NuGet)
% - Vantagens (performance de código nativo, tipagem forte) e desvantagens (curva de aprendizado, tamanho do aplicativo)

\section{Python com PyQt}
\label{sec:python_pyqt}

Python é uma linguagem de programação interpretada de alto nível, conhecida por sua sintaxe clara e legível. Embora não seja tradicionalmente focada em aplicações de interface gráfica, diversas bibliotecas permitem seu uso para este fim. PyQt é um dos bindings mais populares para o framework Qt, uma biblioteca de componentes de UI multiplataforma escrita em C++...
% Aqui você desenvolverá o texto sobre Python com PyQt, abordando:
% - Características da linguagem Python (tipagem dinâmica, GIL)
% - O que é o framework Qt e o papel do PyQt
% - Ecossistema (pip, PyInstaller para distribuição)
% - Vantagens (desenvolvimento rápido, simplicidade) e desvantagens (desempenho em UI, distribuição de apps)

